\section*{Abstract}
The primary goal in detecting anomalies using machine learning in the context of a industry is to develop a model able to detect whether an item is defective or not using data related to said item. This model would be a major improvement for the industry since it takes time to manually test every single part.\\
As expected from production lines' data, defective and non-defective pieces are not equally represented. Since the imbalance between classes harm the model's learning process, balancing them is one way to deal with this issue.\\
The Naïve Bayes classifier was chosen to answer classify items on the production line because it was evaluated as the most performant model at this task among various other machine learning models. The results demonstrate that such a simple algorithm can already predict defects, which is encouraging for the future when more data is collected and heavier models are implemented.